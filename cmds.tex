\DeclareGraphicsRule{.tif}{png}{.png}{`convert #1 `dirname #1`/`basename #1 .tif`.png}


 \definecolor{Comment}{rgb}{0.4,0.2,0.6} % Purple
 \definecolor{Keyword}{cmyk}{0.64,0,0.95,0.40} % Green
 %\definecolor{Constant}{cmyk}{0.62,0.57,0.23,0} % Cadet Blue
 \definecolor{Type}{rgb}{0.9,0.3,0.3} % Pink
 \definecolor{Constant}{rgb}{0.2,0.2,0.9} % Blue

\newcommand{\plumbing}{Plumbing\xspace}

\newcommand{\XXX}{{\em TO BE WRITTEN}\xspace}

\newcommand{\GOALS}{\section{Goals}}
\newcommand{\CODE}{\section{Code}}
\newcommand{\PATTERNS}{\section{Patterns}}
\newcommand{\BREAKAGE}{\section{Breakage}}
\newcommand{\EXPLORATIONS}{\subsection{Explorations}}

\newcommand{\code}{\ttfamily}
\newcommand{\constant}{\ttfamily\bfseries\color{Constant}}
\newcommand{\procname}{\ttfamily}
\newcommand{\channel}{\ttfamily}
\newcommand{\keyword}{\ttfamily\bfseries\color{Keyword}}
\newcommand{\type}{\ttfamily\bfseries\color{Type}}

\newcommand{\PROCedure}{{\keyword PROC}edure\xspace}
\newcommand{\PROC}{{\keyword PROC}\xspace}
\newcommand{\PARallel}{{\keyword PAR}allel\xspace}
\newcommand{\PAR}{{\keyword PAR}\xspace}
\newcommand{\CHANnel}{{\keyword CHAN}nel\xspace}
\newcommand{\CHAN}{{\keyword CHAN}\xspace}

\newcommand{\SIGNALT}{{\type SIGNAL}\xspace}
\newcommand{\SIGNALV}{{\constant SIGNAL}\xspace}

\newcommand{\LEVELT}{{\type LEVEL}\xspace}
\newcommand{\HIGH}{{\constant HIGH}\xspace}
\newcommand{\LOW}{{\constant LOW}\xspace}

\newcommand{\ohm}{$\tcohm$\xspace}

\newcommand{\pinone}{{\constant 1}\xspace}
\newcommand{\pintwo}{{\constant 2}\xspace}
\newcommand{\pinthree}{{\constant 3}\xspace}
\newcommand{\pinfour}{{\constant 4}\xspace}
\newcommand{\pinfive}{{\constant 5}\xspace}
\newcommand{\pinsix}{{\constant 6}\xspace}
\newcommand{\pinseven}{{\constant 7}\xspace}
\newcommand{\pineight}{{\constant 8}\xspace}
\newcommand{\pinnine}{{\constant 9}\xspace}
\newcommand{\pinten}{{\constant 10}\xspace}
\newcommand{\pineleven}{{\constant 11}\xspace}
\newcommand{\pintwelve}{{\constant 12}\xspace}
\newcommand{\pinthirteen}{{\constant 13}\xspace}


%%%%%%%%%%%%%%%%%%%%%%%%%%%%%%%%%%%%%%%%%%
% Macros for layout/controlling common structures in the text.
\newcommand{\webnote}[2]{\footnote{See \url{#1}\xspace for more information about #2.}}
 
% Definition for typesetting occam source code
\lstdefinelanguage{occam} 
{
	keywords={PROC,PAR,SEQ,WHILE,ALT,CHAN,LEVEL},
	keywordstyle=\color{Keyword}\bfseries,
	emph={TRUE, FALSE, LOW, HIGH}, %SIGNAL?
	emphstyle=\color{Constant}\bfseries, 
	emph={[2]INT, BYTE}, %SIGNAL?
	emphstyle={[2]\color{Type}},
	sensitive=true,
	comment=[l]{--},
	string=[b]",
	frame=ltrb,
	framesep=5pt,
  commentstyle=\color{Comment},
  % stringstyle=\ttfamily,
  showstringspaces=true
}

\parskip 2mm


% Layout of source code in general                                    
% Should we use pcr or cmtt for the font? lcmtt? txtt? cmvtt
\lstset{xleftmargin=14pt,basicstyle=\small\fontfamily{lcmtt}\selectfont,captionpos=b,numbers=left,numbersep=10pt,language=occam}

\renewcommand\lstlistingname{Figure}

\newcommand{\mypi}{$\pi$}
\newcommand{\myPI}{\large$\pi$}
\newcommand{\occam}{{\fontfamily{phv}\selectfont occam-\myPI}\xspace}
\newcommand{\occamone}{{\fontfamily{phv}\selectfont occam1}\xspace}
\newcommand{\justoccam}{{\fontfamily{phv}\selectfont occam}\xspace}
\newcommand{\loccam}{{\fontfamily{phv}\selectfont loccam}}
\newcommand{\inmos}{{\fontfamily{phv}\selectfont inmos}\xspace}

%\newcommand{\kroc}{{\fontfamily{phv}\selectfont soccam}}
\def\kroc{KR{\sffamily o}C\xspace} 
\newcommand{\tvm}{Transterpreter\xspace}
\newcommand{\Tvm}{Transterpreter\xspace}


\newcommand{\tvmurl}{\url{www.transterpreter.org}\xspace}
\newcommand{\tvmcite}{\cite{transterpreter}}

\newcommand{\ccc}{\url{www.concurrency.cc}\xspace}


\newcommand{\strong}{\bfseries}

%%%%%%%%%%%%%%%%%%%%%%%%%%%%%%%%%%%%%%%%%%%%%
% Process names
\newcommand{\heartbeat}{{\procname heartbeat}\xspace}
\newcommand{\blink}{{\procname blink}\xspace}
\newcommand{\tp}{{\procname pin.toggle}\xspace}
\newcommand{\bp}{{\procname button.press}\xspace}
\newcommand{\toggle}{{\procname toggle}\xspace}
\newcommand{\digo}{{\procname digital.output}\xspace}
\newcommand{\il}{{\procname invert.level}\xspace}
\newcommand{\tick}{{\procname tick}\xspace}
\newcommand{\pinlevel}{{\procname pin.level}\xspace}

% For cover page
\newcommand{\HRule}{\rule{\linewidth}{0.5mm}}

% Drawing Resistors
\newcommand{\resistor}[3]{\colorbox{#1}{ }\colorbox{#2}{ }\colorbox{#3}{ }\colorbox{Goldenrod}{ }\xspace}


% I'm really unclear regarding left/right on this document.
% I feel like I should have a bigger gutter, but it looks like
% it pushes content *closer* to the gutter. This is wrong.

%\usepackage{fancyhdr}
%\setlength{\headheight}{15pt}
 
%\pagestyle{fancy}
%\renewcommand{\chaptermark}[1]{\markboth{#1}{}}
%\renewcommand{\sectionmark}[1]{\markright{#1}{}}
 
%\fancyhf{}
%\fancyhead[LE,RO]{\thepage}
%\fancyhead[RE]{\textit{\nouppercase{\leftmark}}}
%\fancyhead[LO]{\textit{\nouppercase{\rightmark}}}
 
%\fancypagestyle{plain}{ %
%\fancyhf{} % remove everything
%\renewcommand{\headrulewidth}{0pt} % remove lines as well
%\renewcommand{\footrulewidth}{0pt}}
